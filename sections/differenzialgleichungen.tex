\section{Differenzialgleichungen}


\subsection{Die Differenzialgleichung}

Eine Gleichung zwischen einer Funktion und ihren Ableitungsfunktionen heisst
\emph{Differenzialgleichung}. Die höchste vorkommende Ableitung heisst die
\emph{Ordnung} der Differenzialgleichung. Eine Differenzialgleichung hat in der
Regel unendlich viele Lösungen. Die Lösungsmenge nennt man aus historischen
Gründen auch die \emph{allgemeine Lösung} und eine einzelne Lösung eine
\emph{spezielle Lösung}.  Zusätzliche Bedingungen, welche dazu dienen, aus der
allgemeinen Lösung eine spezielle auszusondern, heissen \emph{Randbedingungen}
oder \emph{Anfangsbedingungen} (wenn das Argument die Zeit darstellt). Eine
Differenzialgleichung mit Randbedingungen wird auch als \emph{Randwertproblem}
oder \emph{Anfangswertproblem} bezeichnet.

\textbf{Merke:} Angewandte Probleme, bei denen eine ganze Funktion gesucht ist,
führen auf Differenzialgleichungen.


\subsubsection{Explizite Differenzialgleichungen}

Eine Differenzialgleichung, in der die höchste Ableitung isoliert auf einer
Seite des Gleichheitszeichens steht, heisst \emph{explizit}. Eine explizite
Differenzialgleichung für die Funktion $f$ hat also die Gestalt
$$f^{(n)}(x) = G\left(x,f(x),f'(x),\dotsc,f^{(n-1)}(x)\right)$$
im Spezialfall einer Differenzialgleichung 1. Ordnung somit
$$f'(x) = G(x,f(x))$$
In diesem Fall ist die rechte Seite ist also nur vom Argument $x$ und vom Wert
$fx$ der gesuchten Funktion abhängig; die Ableitung hingegen kommt rechts nicht
vor.


\subsubsection{Separierbare Differenzialgleichung}

Eine Differenzialgleichung für die Funktion $f$ heisst \emph{separierbar}, wenn
sie auf die Form
$$f'(x) = \frac{g(x)}{h(f(x))}$$
gebracht werden kann.


\subsubsection{Lineare Differenzialgleichung}

Eine Differenzialgleichung für die Funktion $f$ der Form
$$a_0 f(x) + a_1 f'(x) + a_2 f''(x) + \dotsb + a_n f^{(n)}(x) = s(x)$$
oder
$$\sum\limits_{k=0}^n a_k f^{(k)}(x) = s(x)$$
wobei die $a_k$ gegebene Konstanten und $s$ eine vorgegebene Funktion ist,
heisst \emph{lineare Differenzialgleichung} mit konstanten Koeffizienten. Die
$a_k$ heissen die \emph{Koeffizienten}, $s$ die \emph{Störfunktion}.

Wenn die Störfunktion die Nullfunktion ist, heisst die Differenzialgleichung
\emph{homogen}, andernfalls \emph{inhomogen}.


\subsection{Methode von Euler}

Gegeben sei eine explizite Differenzialgleichung 1. Ordnung für die Funktion $f$
$$f'(x) = G(x, y) = G(x,f(x))$$
und ein Punkt $P_n = (x_n,y_n)$ des Graphen einer speziellen Lösung.

Der Punkt $P_{n+1} = (x_{n+1}, y_{n+1})$ wird wie folgt bestimmt:\\
Man wählt ein $\Delta{x}$ mit kleinem Betrag und berechnet
$$x_{n+1} = x_n + \Delta x$$
$$y_{n+1} = y_n + s_n\Delta x$$
mit
$$s_n = G(x_n,y_n)$$


\subsection{Verfahren von Runge-Kutta}

Gegeben sei eine explizite Differenzialgleichung 1. Ordnung für die Funktion $f$
$$f'(x) = G(x, f(x))$$
und ein Punkt $P_n = (x_n, y_n)$ des Graphen einer speziellen Lösung.

Der Punkt $P_{n+1} = (x_{n+1},y_{n+1})$ wird wie folgt bestimmt:\\
Man wählt ein $\Delta{x}$ mit kleinem Betrag. Die Abszisse von $P_{n+1}$ ist
$$x_{n+1} = x_n + \Delta x$$
Die Ordinate ist
$$y_{n+1} = y_n + s\Delta x$$
mit
$$s = \frac{s_1 + 2s_2 + 2s_3 + s_4}{6}$$
wobei
\begin{enumerate}
    \item $\displaystyle s_1 = G(x_n, y_n)$
    \item $\displaystyle s_2 = G\left(x_n + \frac{\Delta{x}}{2}, y_n + s_1
        \frac{\Delta x}{2}\right)$
    \item $\displaystyle s_3 = G\left(x_n + \frac{\Delta{x}}{2}, y_n + s_2
        \frac{\Delta x}{2}\right)$
    \item $\displaystyle s_4 = G(x_n+\Delta x, y_n+s_3\Delta x)$
\end{enumerate}


\subsection{Linearität der Lösungen homogener linearer Differenzialgleichungen}

\begin{enumerate}
    \item Wenn $f_1$ und $f_2$ Lösungen einer homogenen linearen
        Differenzialgleichung sind, so ist auch ihre Summe $f_1 + f_2$ eine
        Lösung.
    \item Ist ferner $f$ eine Lösung und $c$ eine Konstante, so ist auch das
        Produkt $c \cdot f$ eine Lösung.
\end{enumerate}

Man kann die beiden Gesetze auch so zusammenfassen:

Wenn $f_1$ und $f_2$ Lösungen einer homogenen linearen Differenzialgleichung
sind, so ist auch jede Linearkombination $c_1f_1 + c_2f_2$, wobei $c_1$ und
$c_2$ Konstanten sind, eine Lösung.


\subsection{Allgemeine Lösung homogener Linearer Differenzialgleichungen
    zweiter Ordnung}

Gegeben sei die homogene lineare Differenzialgleichung
$$c_2f''(x) + c_1f'(x) + c_0f(x) = 0$$
Um sie zu lösen, formuliert man zuerst ihre \emph{charakteristische Gleichung}.
$$c_2s^2 + c_1s + c_0 = 0$$
Nun sind drei Fälle zu unterscheiden:

\begin{itemize}
    \item Wenn die charakteristische Gleichung \textbf{zwei verschiedene}
        Lösungen $s_1$ und $s_2$ besitzt (die Diskriminante $b^2 - 4ac$ also
        positiv ist), so lautet die allgemeine Lösung der Differenzialgleichung
        $$f := x \mapsto Ae^{s_1 x} + Be^{s_2 x}$$
        mit frei wählbaren Konstanten $A$ und $B$.
    \item Wenn die charakteristische Gleichung \textbf{nur eine} Lösung $s_1$
        besitzt (die Diskriminante $b^2 - 4ac$ also Null ist), so lautet die
        allgemeine Lösung der Differenzialgleichung
        $$f := x \mapsto (Ax + B)e^{sx}$$
        mit frei wählbaren Konstanten $A$ und $B$.
    \item Wenn die charakteristische Gleichung keine Lösung hat (die
        Diskriminante $b^2 - 4ac$ also negativ ist) so lautet die allgemeine
        Lösung der Differenzialgleichung
        $$f := x \mapsto e^{rx}\left(A \cos(\omega x)
            + B \sin(\omega x)\right)$$
        mit frei wählbarem $A$ und $B$, respektive in der Amplituden-Phasen-Form
        $$f := x \mapsto Ce^{rx} \cos(\omega x - \phi)$$
        mit frei wählbarem $C$ und $\phi$.

        Um die Parameter $r$ und $\omega$ zu bestimmen, bringt man die rechte
        Seite der Lösungsformel für die charakteristische Gleichung auf die Form
        $$s = a \pm \sqrt{b}$$
        Dann ist
        $$r = a \textrm{ und } \omega = \sqrt{-b}$$
\end{itemize}


\subsection{Inhomogene lineare Differenzialgleichungen}

Die allgemeine Lösung einer inhomogenen linearen Differenzialgleichung für $f$
$$\sum_{k=0}^n a_k f^{(k)}(x) = s(x)$$
ist die Summe einer partikulären Lösung und der allgemeinen Lösungen der
zugehörigen homogenen Gleichung
$$\sum_{k=0}^n a_k f^{(k)}(x) = 0$$
