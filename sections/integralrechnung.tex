\section{Integralrechnung}


\subsection{Definition des Integrals}

Die Definition des Integrals lautet
$$\int\limits_a^b f = \lim_{n \mapsto \infty}\left(\sum_{i=1}^n f(x_i) \cdot
    \Delta x\right)$$
mit
$$\Delta x = \frac{b-a}{n}$$
und
$$x_i = a + i \cdot \Delta x$$


\subsection{Summenformeln}

$$\sum_{i=1}^n i = \frac{n(n+1)}{2}$$
$$\sum_{i=1}^n i^2 = \frac{n(n+1)(2n+1)}{6}$$
$$\sum_{i=1}^n i^3 = \left(\frac{n(n+1)}{2}\right)^2$$


\subsection{Graphische Interpretation von Integralen}

Wir betrachten das Integral
$$\int\limits_a^b f$$
Wir nennen die Fläche, welche horizontal durch zwei Abszissen und vertikal
durch die Abszissenachse und den Funktionsgraphen begrenzt sind als
\emph{Fläche unter dem Funktionsgraphen}. Es sind nun zwei Fälle zu
unterscheiden:

\begin{itemize}
    \item Wenn $a < b$ ist:\\
    Dann sind Flächen unter Funktionsgraphen mit positiver Ordinate positiv und
    solche mit negativen Ordinaten negativ zu zählen.
    \item Wenn $a > b$ ist:\\
    Dann sind Flächen unter Funktionsgraphen mit positiver Ordinate negativ und
    solche mit negativen Ordinaten positiv zu zählen.
\end{itemize}


\subsection{Vorzeichenregeln und Additivität}

Falls die beteiligten Integrale existieren, gilt
\begin{itemize}
    \item Vertauschen der Integralgrenzen ändert das Vorzeichen des Integrals
    $$\int\limits_a^b f = - \int\limits_b^a f$$
    \item Aneinanderstossende Integrale können zusammengefasst werden.
    $$\int\limits_a^b f + \int\limits_b^c f = \int\limits_a^c f$$
\end{itemize}


\subsection{Numerische Berechnung von Integralen}


\subsubsection{Trapezregel}

Die allgemeine Formel für die Trapezregel lautet:

$$\int\limits_a^b f = \left[\frac{f(a) + f(b)}{2} + \sum_{i=1}^{n-1}
    f(x_i)\right] \Delta x$$
mit
$$\Delta x = \frac{b-a}{n} \textrm{ und } x_i = a + i \cdot \Delta x$$


\subsubsection{Fassregel}

Bei der Fassregel wird die zu integrierende Funktion $\int_a^b$ mit einer
quadratischen Funktion, die durch die Punkte $a$, $b$ und $\frac{b-a}{2}$ geht,
approximiert.
$$\int\limits_a^b f \approx \int\limits_a^b q = \frac{y_a + 4y_m
    + y_b}{3} \Delta x$$
wobei
$$y_a = f(a) \textrm{, } y_m = f\left(\frac{a+b}{2}\right) \textrm{, } y_b
    = f(b) \textrm{, } \Delta x = \frac{b-a}{2}$$


\subsubsection{Simpson-Regel}

Sei $f$ eine auf $[a;b]$ viermal differenzierbare Funktion und $n$ eine gerade
Zahl. Ferner sei
$$x_i = a + i \cdot \Delta x \textrm{ mit } \Delta x = \frac{b-a}{n}
    \textrm{ und } y_i = f(x_i)$$
Dann ist
$$S_n = \frac{\Delta x}{3}(y_0 + 4y_1 + 2y_2 + 4y_3 + 2y_4 + ... + 4y_{n-1}
    + y_n)$$
$$= \frac{\Delta x}{3}\left(y_0 + y_n + 4 \sum_{k=1}^{n/2} y_{2k-1} + 2
    \sum_{k=1}^{n/2-1} y_{2k}\right)$$
eine Schätzung für das Integral $\int\limits_a^b f$, wobei der Fehler
$$E_n = \left(\int\limits_a^b f\right) - S_n = \frac{b-a}{180}\Delta x^4
    f^{(4)}(\xi) = \frac{(b-a)^5}{180n^4} f^{(4)}(\xi)$$
$$\textrm{für ein } \xi \in [a;b]$$
beträgt.


\subsection{Integralfunktion}

Gegeben sei eine auf dem Intervall $[a;b]$ integrierbare Funktion $f$. Dann
heisst jede Funktion der Form
$$x \mapsto \int\limits_c^x f$$
für eine Konstante $c \in [a;b]$ eine \emph{Integralfunktion} von $f$.


\subsection{Zusammenhang verschiedener Integralfunktionen}

Verschiedene Integralfunktionen derselben Funktion unterscheiden sich nur durch
eine Konstante. Wenn also
$$\varphi_c := x \mapsto \int\limits_c^x f \textrm{ und } \varphi_d := x
    \mapsto \int\limits_d^x f$$
dann gilt
$$\varphi_d = \varphi_c + k \textrm{ wobei } k \textrm{ konstant}$$


\subsection{Hauptsatz der Infinitesimalrechnung}

Die Ableitungsfunktion einer Integralfunktion einer stetigen Funktion ist gleich
der integrierten Funktion.
$$\left(x \mapsto \int\limits_a^x f\right)' = f$$


\subsection{Stammfunktion}

Sei $f$ eine reelle Funktion. Wenn es eine Funktion $F$ gibt, so dass $f$ ihre
Ableitungsfunktion ist, also
$$F' = f$$
gilt, dann heisst $F$ eine \emph{Stammfunktion} von $f$.


\subsection{Unbestimmtes Integral}

Die Menge aller Stammfunktionen einer Funktion $f$ nennt man ihr
\emph{unbestimmtes Integral} und bezeichnet es mit
$$\int f \textrm{ (ohne Grenzen)}$$
Auch von Termen gibt es das unbestimmte Integral. Wenn etwa $T$ ein Term und
$x$ eine Variable ist, schreibt man
$$\int Tdx$$


\subsection{Integralfunktion - Stammfunktion}

Jede Integralfunktion einer stetigen Funktion ist eine Stammfunktion von ihr.


\subsection{Berechnung von Integralen mit Hilfe von Stammfunktionen}

Sei $f$ eine auf den Intervall $[a;b]$ \textbf{stetige} Funktion und $F$ eine
Stammfunktion von ihr. Dann gilt
$$\int\limits_a^b f = F\Big |_a^b = F(b) - F(a)$$


\subsection{Integrationsregeln}


\subsubsection{Linearitätsregel}

Seien $f$ und $g$ Funktionen und $c$ eine Konstante. Dann gelten folgende
Linearitätsregeln:
$$\int (f+g) = \int f + \int g$$
$$\int cf = c \int f$$


\subsubsection{Integrale von Verkettungen mit linearen Funktionen}

Sei $f$ eine integrierbare Funktion und $F$ eine Stammfunktion von ihr. Dann
gilt
$$\int f(ax+b)dx = \frac{F(ax+b)}{a} + c$$
wobei $c$ eine beliebige Konstante ist.


\subsubsection{Produktregel}

$$\int\limits_a^b f(x) \cdot g'(x) dx = (f(x) \cdot g(x))\Big |_a^b
    - \int f'(x) \cdot g(x) dx$$


\subsubsection{Produkt mit der eigenen Ableitung} 

Eine Stammfunktion des Produktes einer Funktion mit ihrer eigenen
Ableitungsfunktion ist gleich dem halben Quadrat dieser Funktion.

Beispiel:
$$\int \sin(x) \cos(x) dx = \frac{\sin(x)^2}{2} + c$$


\subsubsection{Quotientenregel}

$$\int\limits_a^b \frac{f'(x)}{f(x)} dx = \ln(|f(x)|)\Big |_a^b$$


\subsubsection{Substitutionsregel}

$$\int\limits_a^b f(g(x)) \cdot g'(x) dx = \int\limits_{g(a)}^{g(b)} f(u) du$$


\subsection{Fläche zwischen Funktionsgraphen} 

$f$ und $g$ seien Funktionen, die im Intervall $[a;b]$ stetig sind. Dann ist
die Fläche, welche oben und unten durch die Graphen der Funktionen $f$ und $g$,
sowie links und rechts durch die Abszissen $a$ respektive $b$ begrenzt wird,
durch
$$\int\limits_a^b |f(x) - g(x)| dx$$
gegeben.
